% Created 2019-05-27 Mon 18:08
% Intended LaTeX compiler: pdflatex
\documentclass[11pt]{article}
\usepackage[utf8]{inputenc}
\usepackage[T1]{fontenc}
\usepackage{graphicx}
\usepackage{grffile}
\usepackage{longtable}
\usepackage{wrapfig}
\usepackage{rotating}
\usepackage[normalem]{ulem}
\usepackage{amsmath}
\usepackage{textcomp}
\usepackage{amssymb}
\usepackage{capt-of}
\usepackage{hyperref}
\usepackage[a4paper, margin=2cm]{geometry}
\usepackage{indentfirst}
\usepackage[, brazilian]{babel}
\usepackage{float}
\usepackage{color, colortbl}
\usepackage{titling}
\setlength{\droptitle}{-1.5cm}
\hypersetup{ colorlinks = true, urlcolor = blue }
\definecolor{beige}{rgb}{0.93,0.93,0.82}
\definecolor{brown}{rgb}{0.4,0.2,0.0}
\date{}
\title{}
\hypersetup{
 pdfauthor={},
 pdftitle={},
 pdfkeywords={},
 pdfsubject={},
 pdfcreator={Emacs 26.2 (Org mode 9.2)},
 pdflang={Brazilian}}
\begin{document}

\tableofcontents


\section{Memória virtual - Desempenho (aula 27/05)}
\label{sec:orgf8a860c}
\subsubsection{Otimização}
\label{sec:org86e758c}
Se encher a memória, é necessária uma política de reposição de páginas. A página só é
escrita no disco caso já se tenha escrito nele.

\subsubsection{Frames}
\label{sec:orgcd24344}
\begin{itemize}
\item Frame: página física
\item Page: página virtual
\end{itemize}

Lugares na memória principal onde colocar as pages.

Dado uma sequênca de acessos, como o algoritmo se comporta?
\begin{enumerate}
\item Aleatoriamente;
\item Monitoração de execução de programas reais
\end{enumerate}

\subsubsection{FIFO}
\label{sec:org442e27b}
Retira-se a página que está a mais tempo na memória. Anomalia de \textbf{belady}.

\subsection{Fifo e segunda chance}
\label{sec:orgc90cc91}
\begin{enumerate}
\item Segunda chance: FIFO modificado. Se o bit de referência for 0, remova-a. Se for 1,
dê uma segunda chance. Fácil de implementar, degenera se todos os bits estiverem
setados. Percorre a lista de todas, não pega a mais recente. Estão todos setados
caso a memória esteja com muita demanda.
\item Segunda chance ao segunda chance: páginas na ordem de 1 a 4. De sem ref e limpa a
com ref e suja.
\end{enumerate}

\subsubsection{Política ótima}
\label{sec:orge6106c5}
Não factível! Pode ser aproximada. Número de page faults é menor.

\subsubsection{LRU}
\label{sec:orga23b587}
Remove a página que foi usada há mais tempo. Implementação complicada, pois tem que
guardar cada posição de memória e seu tempo! Não sofre da anomalia de Belady!
\begin{enumerate}
\item Aproximações
\label{sec:org89ee822}
\begin{enumerate}
\item \textbf{LRU:} Basta inicializar o bit de referência com 0 e deixar o hardware colocá-lo em 1 quando
a página for acessada. E quando todas as páginas tiverem sido referenciadas?
\item \textbf{LRU:} mais bits de referência: Guardar o número de bits em um byte. SHR e \&. A página com o menor valor do byte
de referência deve ser removida.
\end{enumerate}
\end{enumerate}


\subsubsection{LFU e MFU = implementações caras.}
\label{sec:org3d33f2a}

\subsection{Modelo de working set}
\label{sec:org107cd58}
Janela de trabalho é uma medida de localidade. \(\Delta\) é seu tamanho. É a medida das
últimas delta references a páginas da memória.

O tamanho médio da janela é o número de frames a serem alocados por processo.

\subsubsection{Como definir \(\Delta\)?}
\label{sec:orgb17268a}
Monitoração de um conjunto de programas.

\begin{enumerate}
\item Formalmente Não permite a utilização de mais que \emph{n} frames.
\item \textbf{Informalmente} (melhor jeito) Assume-se que o processo vai usar \emph{n} frames, mas não se garante.
Serve para ter uma ideia de quantos processos o sistema consegue tratar de forma eficiente.
\end{enumerate}

\subsection{Thrasing}
\label{sec:org2b1a2fb}
Solução para o caso de haverem mais processos.

Número de page faults aumenta consideravelmente. Quando o pc para e não responde mais
(gasta-se mais tempo decidindo o que fazer, do que fazendo algo). \textbf{Ex:} ineficência de
plugins de navegadores.
\end{document}
