% Created 2019-06-10 Mon 17:14
% Intended LaTeX compiler: pdflatex
\documentclass[11pt]{article}
\usepackage[utf8]{inputenc}
\usepackage[T1]{fontenc}
\usepackage{graphicx}
\usepackage{grffile}
\usepackage{longtable}
\usepackage{wrapfig}
\usepackage{rotating}
\usepackage[normalem]{ulem}
\usepackage{amsmath}
\usepackage{textcomp}
\usepackage{amssymb}
\usepackage{capt-of}
\usepackage{hyperref}
\usepackage[a4paper, margin=2cm]{geometry}
\usepackage{indentfirst}
\usepackage[, brazilian]{babel}
\usepackage{float}
\usepackage{color, colortbl}
\usepackage{titling}
\setlength{\droptitle}{-1.5cm}
\hypersetup{ colorlinks = true, urlcolor = blue }
\definecolor{beige}{rgb}{0.93,0.93,0.82}
\definecolor{brown}{rgb}{0.4,0.2,0.0}
\date{}
\title{}
\hypersetup{
 pdfauthor={},
 pdftitle={},
 pdfkeywords={},
 pdfsubject={},
 pdfcreator={Emacs 26.2 (Org mode 9.2)},
 pdflang={Brazilian}}
\begin{document}

\tableofcontents


\section{Memória virtual - Desempenho (aula 27/05)}
\label{sec:org7739ce3}
\subsubsection{Otimização}
\label{sec:orgd45a618}
Se encher a memória, é necessária uma política de reposição de páginas. A página só é
escrita no disco caso já se tenha escrito nele.

\subsubsection{Frames}
\label{sec:org5de2f79}
\begin{itemize}
\item Frame: página física
\item Page: página virtual
\end{itemize}

Lugares na memória principal onde colocar as pages.

Dado uma sequênca de acessos, como o algoritmo se comporta?
\begin{enumerate}
\item Aleatoriamente;
\item Monitoração de execução de programas reais
\end{enumerate}

\subsubsection{FIFO}
\label{sec:org2dec41c}
Retira-se a página que está a mais tempo na memória. Anomalia de \textbf{belady}.

\subsection{Fifo e segunda chance}
\label{sec:orgfcc7100}
\begin{enumerate}
\item Segunda chance: FIFO modificado. Se o bit de referência for 0, remova-a. Se for 1,
dê uma segunda chance. Fácil de implementar, degenera se todos os bits estiverem
setados. Percorre a lista de todas, não pega a mais recente. Estão todos setados
caso a memória esteja com muita demanda.
\item Segunda chance ao segunda chance: páginas na ordem de 1 a 4. De sem ref e limpa a
com ref e suja.
\end{enumerate}

\subsubsection{Política ótima}
\label{sec:orgf8878e6}
Não factível! Pode ser aproximada. Número de page faults é menor.

\subsubsection{LRU}
\label{sec:org3510b92}
Remove a página que foi usada há mais tempo. Implementação complicada, pois tem que
guardar cada posição de memória e seu tempo! Não sofre da anomalia de Belady!
\begin{enumerate}
\item Aproximações
\label{sec:org7e7cf92}
\begin{enumerate}
\item \textbf{LRU:} Basta inicializar o bit de referência com 0 e deixar o hardware colocá-lo em 1 quando
a página for acessada. E quando todas as páginas tiverem sido referenciadas?
\item \textbf{LRU:} mais bits de referência: Guardar o número de bits em um byte. SHR e \&. A página com o menor valor do byte
de referência deve ser removida.
\end{enumerate}
\end{enumerate}


\subsubsection{LFU e MFU = implementações caras.}
\label{sec:orgc0a8c6b}

\subsection{Modelo de working set}
\label{sec:orgba59e4c}
Janela de trabalho é uma medida de localidade. \(\Delta\) é seu tamanho. É a medida das
últimas delta references a páginas da memória.

O tamanho médio da janela é o número de frames a serem alocados por processo.

\subsubsection{Como definir \(\Delta\)?}
\label{sec:org62bf2ad}
Monitoração de um conjunto de programas.

\begin{enumerate}
\item Formalmente Não permite a utilização de mais que \emph{n} frames.
\item \textbf{Informalmente} (melhor jeito) Assume-se que o processo vai usar \emph{n} frames, mas não se garante.
Serve para ter uma ideia de quantos processos o sistema consegue tratar de forma eficiente.
\end{enumerate}

\subsection{Thrasing}
\label{sec:org9517248}
Solução para o caso de haverem mais processos.

Número de page faults aumenta consideravelmente. Quando o pc para e não responde mais
(gasta-se mais tempo decidindo o que fazer, do que fazendo algo). \textbf{Ex:} ineficência de
plugins de navegadores.
\section{Memória virtual - Linux (aula 29/05)}
\label{sec:orgfcc4a05}
Compartilhamento de memória permite um ganho significativo no desempenho.

No Linux podemos identificar compartilhamento de memória:
\begin{itemize}
\item Entre buffers de bloco e página;
\item Entre páginas através de copy-on-write;
\item Entre subrotinas sendo executadas.
\end{itemize}
Foco na memória principal.

\subsection{Entre buffers}
\label{sec:org97e474e}
\subsubsection{Buddy heap}
\label{sec:org6a7b583}
Páginas são agrupadas em blocos de dois por listas encadeadas.
\subsubsection{Contadores de referência}
\label{sec:orga698623}
Usados para saber se alguma página já foi usada e pode ser liberada ou se ainda existe
alguém que precise dela.

Inicia-se com c=0. Quando cria-se uma página, c\(++\); quando é destruída, c\(--\).

Simples e eficiente.

\subsection{Entre páginas}
\label{sec:org1fbc7ff}
\subsubsection{Diversos sabores de páginas}
\label{sec:orga602381}
\begin{itemize}
\item Sem pistolão: zero filled (primeiro acesso retorna página vazia).
\item Com pistolão: backed (acessos à página retornam páginas de arquivo padrinho).

\item Private: acesso por único processo.
\item Shared: acesso por vários processos.
\end{itemize}

\subsubsection{Fork}
\label{sec:orgb2b57ec}
Copia todas as páginas na memória para a nova tarefa. Ineficiente e problemas de
acesso.

\begin{enumerate}
\item Copy-on-write (quase máximo compartilhamento)
\label{sec:org142982e}
Compartilha a página somente na leitura.
\end{enumerate}

\subsection{Subrotinas}
\label{sec:org67c1a49}

\subsubsection{a.out}
\label{sec:org1008bb4}
Código todo é linkado automaticamente.

\subsubsection{ELF (mais moderno)}
\label{sec:org467f139}
   Linkado dinamicamente (carrega subrotinas sob demanda). \textbf{Shared libs:} se a subrotina
estiver na memória, o SO não carrega outra cópia.
\section{Arquivos (aula 03/06)}
\label{sec:orgbc7f722}
\subsection{Definição}
\label{sec:org18e933e}
Conjunto de dados normalmente não voláteis; menor unidade de informação secundária.
Qualquer dado não-volátil tem que ser guardado em arquivos.
\subsection{Definido por}
\label{sec:orged7baef}
\begin{itemize}
\item Nome
\item Tipo
\item Lugar
\item Tamanho
\item Permissões
\item Tempos
\end{itemize}
\subsection{O que mais além de arquivos?}
\label{sec:org3a7edc9}
\begin{itemize}
\item Diretórios: coleção de arquivos.
\item Partições: pedaços de memória secundária
\end{itemize}
\subsection{Operações com arquivos}
\label{sec:orgbbc4828}
Criar, ler, mover, reposicionar(bem útil), truncar (cortar de um ponto para frente), apagar.
\subsection{Abir/fechar}
\label{sec:org3553f0d}
Não é essencial, apenas uma otimização através de buffers na memória
principal. Consegue-se fazer leitura sem abrir o arquivo.
\subsubsection{Problemas}
\label{sec:orge83d8d4}
\begin{itemize}
\item Esquecer de fechar (memory leak)
\item Acessos concorrentes
\end{itemize}
\subsubsection{Tomar cuidado com}
\label{sec:orgf44a715}
\begin{itemize}
\item Localização no disco (pode estar mudando)
\item Reference count pra ter certeza de que ninguém vai apagar coisas que não estão sendo
usadas.
\end{itemize}
\subsection{Métodos de acesso}
\label{sec:org025284f}
\begin{itemize}
\item Sequencial: ler, escrever e rebobinar (originado de fita). Ex: pdf, áudio, zip, tar,
.o.
\item Direto: ler, escrever, reposicionar (seek) - originado de disco. Ex: banco de dados,
binário, libraries, ávores B.
\end{itemize}
\subsection{Diretórios}
\label{sec:orgfaf4081}
Estrutura é um DAG (grafo acíclico direcionado)
\begin{itemize}
\item Cria arquivo
\item Apaga
\item Busca
\item Lista conteúdo
\end{itemize}

Não é muito diferente de arquivos. Symlink(shortcut) e hard link().

Append só serve para log.
\section{Arquivos - Implementação (aula 03/06)}
\label{sec:org6deddee}
\subsection{Alocação de espaço em disco}
\label{sec:orgbc43606}
\subsubsection{Alocação contígua}
\label{sec:org53eccab}
Fácil implementação e rápido acesso. Não é muito usado.
\subsubsection{Encadeadas}
\label{sec:org7d7b71e}
Ex: sistema pick - lista duplamente encadeada (só banco de dados).
\begin{itemize}
\item Sem fragmentação, simples de aumentar, criar e apagar. Aproveita o máximo o disco.
\item Ruim para acesso sequencial. Se um link se perder, perde-se o arquivo todo (corrompe
a informação).

Não é muito usado.
\end{itemize}
\begin{enumerate}
\item FAT
\label{sec:org73ef5fe}
Diminui os erros, menos eficiente.
\end{enumerate}
\subsubsection{Indexada}
\label{sec:orgcceec03}
Uma fat para todos os arquivos; um índice por arquivo. Não importa a eficiência.
\begin{itemize}
\item Tem que usar um índice, mesmo que o arquivo use um bloco só.
\item Tamanho ideal é um bloco de disco.
Solução: unix \emph{inode}.
\end{itemize}
\section{Sistemas de E/S (aula 05/06)}
\label{sec:orgd608111}
\begin{itemize}
\item mouse é lento
\item discos e impressora rápidos
\item trânsito de dados complexo, velocidade difernetes: E/S complexos
\item Sistema embutido é mais simples.
\item Sistemas embutidos são usados em robôs, máquinas, controladores de aeronaves.
\end{itemize}
\subsection{Tipos de dispositivos e/s}
\label{sec:orgabd139c}
Só há duas classificações:
\subsubsection{Bloco}
\label{sec:org7c1daa6}
\begin{itemize}
\item podem ler escrever em qualquer bloco
\item ex: discos
\end{itemize}
\subsubsection{Caracteres}
\label{sec:org32c8e9a}
\begin{itemize}
\item dados são sequência de caracteres
\item não tem seek
\item ex: termiais, impressoras, mice, clock, video, fitas.
\end{itemize}

\subsection{Componentes de um dispositivo}
\label{sec:org05b80ee}
\begin{itemize}
\item Dispositivo
\item Controlador
\end{itemize}
\subsubsection{Software que controla (device driver)}
\label{sec:org6be19fc}
Nem sempre funciona: \(ioctl()\)

\subsection{Processador}
\label{sec:org1da4b15}
É feito um mapeamento do processador para os registradores:
\begin{itemize}
\item através de comandos especiais de leituras de portas;
\item através de mapeamento de memória (mais usado, porém pode levar a erros). Ex associar
um endereço ao disco rígido;
\end{itemize}

\subsection{Modo de acesso}
\label{sec:orgc9ed678}
Usa-se todos, não consegue-se usar DMA para todos (restrito). Usa-se as interrupções
rapidamente, aí tem que usar polling.
\subsubsection{Polling}
\label{sec:org6adf8bc}
Handshake: fica esperando o resultado chegar (já chegou? já chegou?). Espera ocupada.
\subsubsection{Interrupções}
\label{sec:org9a7dfe0}
Sem espera ocupada. Cpu programa E/S. Quando os dados ficam prontos, o cpu dá um jump
(interrupção) e dados são lidos.
\subsubsection{DMA (método mais eficiente)}
\label{sec:org0ddaeb0}
Métodos anteriores possuem transferência de poucos bytes por vez.  Vc tem um
controlador ligado ao barramento. Está executando um programa, aloca um buffer e manda
uma msg pro DMA. DMA guarda as palavras no buffer e gera apenas uma interrupção para
voltar. Copia um bloco inteiro por vez.

\subsection{Software}
\label{sec:org7eb6451}
Objetivo: ser independente do dispositivo.

Interface única. Tratado como arquivo.


\begin{itemize}
\item Blocante ou assíncrono (em geral é)
\item Não blocante
\end{itemize}


Programas são síncronos.

\begin{itemize}
\item Compartilhar (ex: discos)
\item Não compartilhar (ex: impressoras)
\end{itemize}
\section{Segurança (aula 10/06)}
\label{sec:org3928316}
\subsection{Razões de perda de dados}
\label{sec:org3d4b63d}
\begin{itemize}
\item Catástrofe naturais
\item Erros de hw/sw
\item Erros humanos
\item Erros bizantinos (invasores)
\end{itemize}

\subsection{Invasores}
\label{sec:orgfe4e4dc}
\begin{itemize}
\item Passivos: leitura
\item Ativos: modificação
\item Ex: casuais, internos (pertence à organização), externos (), determinados, hackers, ladrões, espionagem, violação
de privacidade.
\end{itemize}
\end{document}
